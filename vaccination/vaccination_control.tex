\documentclass[english,12pt,letter]{article}

\usepackage{amsthm}
\usepackage[latin9]{inputenc}
\usepackage{babel}
\usepackage[hmargin=0.9in,vmargin=1.25in]{geometry}
\usepackage{graphicx}
\usepackage{subfigure}
\usepackage[colorlinks=true,citecolor=blue,urlcolor=blue]{hyperref}
\usepackage{amsmath}
\usepackage{amssymb,amsfonts}

\newtheorem{thm}{Theorem}
\newtheorem{lem}{Lemma}
\newtheorem{cor}{Corollary}
\newtheorem{dfn}{Definition}

\newcommand{\Rnot}{\sigma_0}
\newcommand{\xinf}{x^\infty}
\newcommand{\dom}{{\mathcal D}}
\newcommand{\R}{{\mathbb R}}
\newcommand{\xopt}{x_\text{opt}}
\newcommand{\yopt}{y_\text{opt}}
\newcommand{\ymax}{y_\text{max}}
\newcommand{\xoneinf}{x_1^\infty}
\newcommand{\xtwoinf}{x_1^\infty}
\newcommand{\xonetwoinf}{x_{1,2}^\infty}
\newcommand{\sdummy}{\hat{s}}

\DeclareMathOperator\supp{supp}

\begin{document}
\title{Optimal control of a multi-cohort SIR epidemic through scarce vaccination over time}
\author{
  David I. Ketcheson\thanks{Computer, Electrical, and Mathematical Sciences \& Engineering Division,
King Abdullah University of Science and Technology, 4700 KAUST, Thuwal
23955, Saudi Arabia. (david.ketcheson@kaust.edu.sa)}
}
\maketitle


\section{Problem description and assumptions}
In this work we consider an extension of the
classical SIR model of Kermack \& Mckendrick \cite{kermack1927contribution}, in which
the general population is divided into multiple groups, or cohorts, depending on
factors like risk or geographic location.  We assume that a vaccine becomes
available in limited quantities over time, and consider the problem of choosing
how best to distribute the vaccine among the various cohorts.

The population is divided into susceptible ($x$), infected ($y$), and removed ($z$) fractions.
We assume the epidemic occurs on a timescale over which the the overall population
and the individual cohorts do not change their composition, so that
\begin{align}
    \sum_i (x_i(t) + y_i(t) + z_i(t))dt = 1.
\end{align}
The model takes the form
\begin{subequations} \label{SIR}
\begin{align} 
    x_i'(t) & = -x_i(t) \sum_j \beta_{ij}  y_j(t) - u_i(t) \\
    y_i'(t) & = x_i(t) \sum_j \beta_{ij} y_j(t) - \gamma y_i(t) \label{eq:y} \\
    z_i'(t) & = \gamma y_i(t) + u_i(t) \label{eq:z}.
\end{align}
\end{subequations}
Here $u_i(t)$ are the control variables which indicate the rate of vaccination
among cohort $i$.  The values $\beta_{ij}$ are the statistical contact rates
between the different cohorts.
The total amount of vaccine available up to time $t$ is denoted
by $a(t)$.  Vaccination is constrained by
\begin{align}
    \sum_i \int_0^t u_i(t) dt \le a(t).
\end{align}

Our first objective function is simply to minimize the number of deaths.  Let
$r_i$ denote the infection fatality ratio for cohort $i$.
Then the number of deaths in cohort $i$ up to time $T$ is
\begin{align}
 &  r_i \left(z_i(T) - \int_0^T u_i(t) dt\right) \\
 = & r_i\left(p_i-x_i - \int_0^T u_i(t) dt\right),
\end{align}
where $p_i$ is the total population of cohort $i$.
We thus take as objective function (omitting the constant term $r_i p_i$)
\begin{align}
    J & = - \sum_i r_i \left( x_i(T) + \int_0^T u_i(t) dt\right).
\end{align}
The Hamiltonian is then
\begin{align} \label{hamiltonian}
    H = - \sum_i \lambda_i x_i \sum_j \beta_{ij} y_j + \sum_i \mu_i x_i \sum_j \beta_{ij} y_j 
            - \sum_i \mu_i \gamma y_i - \sum_i (\lambda_i + r_i) u_i.
\end{align}
Here $\lambda_i, \mu_i$ are the adjoint variables, which satisfy
\begin{align}
    \lambda_i'(t) & = -\frac{\partial H}{\partial x_i} = (\lambda_i - \mu_i) \sum_j \beta_{ij} y_j \\
    \mu_i'(t) & = -\frac{\partial H}{\partial y_i} = \sum_j (\lambda_j - \mu_j) \beta_{ji} x_j + \gamma \mu_i \\
    \lambda_i(T) & = - r_i \\
    \mu_i(T) & = 0.
\end{align}
We see immediately from \eqref{hamiltonian} that it is optimal to use the available
vaccine immediately on the cohort with the largest value of $\lambda_i + r_i$,
since a necessary condition for optimiality is pointwise minimization of $H$.
(prove this?)


\section{Extended compartmental model}
Now we consider a model that also includes asymptomatic groups.  The asymptomatic
infected eventually become unknown recovered.
When the vaccine is administered, it is distributed proportionally among the susceptible,
asymptomatic, and unknown recovered groups.  Contact rates for the symptomatic
group are significantly lowered.

\subsection{Single cohort without screening}
The model takes the form
\begin{align}
    S'(t) & = -S(\beta^1 A + \beta^2 I + v) \\
    {A}'(t) & = \alpha S (\beta^1 A + \beta^2 I) - (\gamma_1 + v) A \\
    {I}'(t) & = (1-\alpha) (\beta^1 A + \beta^2 I) - (\gamma_2 + \gamma_3) I \\
    {U}'(t) & = -U v + \gamma_1 A \\
    {R}'(t) & = \gamma_2 I \\
    V'(t) & = v(S + A + U) \\
    D'(t) & = \gamma_3 I.
\end{align}
Here $A, I$ are respectively the asymptomatic and symptomatic individuals;
$U, R$ are their recovered counterparts.  The rate of vaccination (proportional
to the candidate population $S+A+U$) is given by $v$.
Because populations $R$, $D$, and $V$ do not appear on the right hand side of the
system, it is not necessary to explicitly track them in order to solve the optimal control problem.
Omitting these, we are left with a system of 4 equations for $S$, $A$, $I$, and $U$.


\subsection{Screening}
We assume that it is possible to administer a test (e.g. an antibody test) to determine
whether an individual already has natural immunity; i.e. to reveal individuals in the
unknown recovered group and move them to the known recovered group.  The purpose of
this test (referred to as screening) is to increase the efficiency of the vaccination
program.  Here we consider three potential screening strategies.

\subsubsection{No screening}
In this approach, no screening is done and vaccines are administered randomly
among the apparently susceptible group, which consists of $S$, $A$, and $U$.
The number of individuals selected from each group will be proportional
to the size of that group.

The cost to vaccinate a part of the apparently susceptible group is proportional to
$$
c_v (S + A + U),
$$
and the change in the Hamiltonian is proportional to
\begin{align} \label{delta-H}
    - (\lambda_S S + \lambda_{A} A + \lambda_{U} U).
\end{align}

\subsubsection{Random screening}
An approach that is simple to model is the application of random screening
that is not correlated with vaccination.  Screening tests are administered
in the same way as vaccines, to randomly selected individuals from the
apparently susceptible group.  

The cost to screen a part of the apparently susceptible group is proportional to
$$
c_\omega (S + A + U).
$$

This has the advantage that screened individuals in $U$ are moved to $R$,
increasing the effectiveness of future vaccine distribution.
This strategy has the drawback of the additional cost of screening.  Note that
some individuals who are screened would never have
been selected for vaccination anyway.  Hence, those screening costs are wasted.

\subsubsection{Targeted screening}
In this approach, individuals are first selected for vaccination and then screened
prior to receiving the vaccine.  The vaccine is administered only if they test
positive for antibodies.  This is strictly superior to the random screening
approach since no antibody test is wasted.  The cost of the screening plus
vaccination is proportional to
$$
    c_\omega (S + A + U) + c_v S
$$
while the change in the Hamiltonian is again \eqref{delta-H} (since
it makes no difference to the objective whether people are moved to the
known recovered group or the vaccinated group).
Therefore, targeted screening is preferable over no screening when the cost
difference
$$
    c_\omega (S + A + U) - c_v U
$$
is negative.


\subsection{Single cohort with option for targeted screening}
Now we let $\nu_1$ denote random screening and $\nu_2$ denote targeted screening.
The model takes the form
\begin{align}
    S'(t) & = -S(\beta^1 A + \beta^2 I + \nu_1 + \nu_2) \\
    A'(t) & = \alpha (S+L) (\beta^1 A + \beta^2 I) - (\gamma_1 + \nu_1) A \\
    I'(t) & = (1-\alpha) (S+L) (\beta^1 A + \beta^2 I) - (\gamma_2 + \gamma_3 + \nu_1) I \\
    U'(t) & = -U (\nu_1 + \nu_2) + \gamma_1 A \\
    L'(t) & = f(\nu_1+\nu_2)S - L(\beta^1 A + \beta^2 I) \\
    R'(t) & = \gamma_2 I + \nu_2 U \\
    V'(t) & = \nu_1 (S + A + U) \\
    D'(t) & = \gamma_3 I.
\end{align}
The rate of screened vaccination is given by $\nu_2$; the rate of vaccine failure is $f$.
Again, because populations $R$, $D$, and $V$ do not appear on the right hand side of the
system, it is not necessary to explicitly track them in order to solve the optimal control problem.

The objective function is
\begin{align}
    J & = \int_0^T \left( (\xi_D \gamma_3 + c_p) I + c_\nu (\nu_1 (S+A+U) + \nu_2 S) + c_\omega \nu_2 (S+A+U)\right),
\end{align}
where $c_\nu, c_\omega$ are the cost of vaccination and screening, respectively, while $c_p$ is the
combined economic cost of treating someone and the loss of their productivity, and $\xi_D$ is the
relative value of a life.

The Hamiltonian is
\begin{align*}
    H & =  -\lambda_S S (\beta^1 A + \beta^2 I + \nu_1 + \nu_2)
            + \lambda_A ( \alpha(S+L)(\beta^1 A + \beta^2 I) - (\gamma_1+\nu_1) A) \\
      &     + \lambda_I ( (1-\alpha)(S+L)(\beta^1 A + \beta^2 I) - (\gamma_2 + \gamma_3 + \nu_1) I )
            + \lambda_U ( \gamma_1 A - U(\nu_1 + \nu_2) ) \\
      &     + \lambda_L ( fS(\nu_1 + \nu_2) - L(\beta^1 A + \beta^2 I) )
            + (\xi_D \gamma_3 + c_p) I + c_\nu (\nu_1 (S+A+U) + \nu_2 S) + c_\omega \nu_2 (S+A+U).
\end{align*}

The adjoint variables satisfy
\begin{align}
    \lambda_S'(t) = -\frac{\partial H}{\partial S} & = (\lambda_S - \alpha \lambda_A - (1-\alpha) \lambda_I) (\beta^1 A + \beta^2 I) + (\lambda_S - \lambda_L f)(\nu_1+\nu_2) - c_\nu (\nu_1 + \nu_2) - c_\omega \nu_2 \\
    \lambda_A'(t) = -\frac{\partial H}{\partial A} & =  \beta^1 ( S \lambda_S + L\lambda_L - (S+L)(\alpha\lambda_A + (1-\alpha)\lambda_I)) - (\gamma_1+\nu_1)\lambda_A - \gamma_1\lambda_U - c_\nu \nu_1 - c_\omega \nu_2 \\
    \lambda_I'(t) = -\frac{\partial H}{\partial I} & = \beta^2(S\lambda_S + L\lambda_L - (S+L)(\alpha \lambda_A + (1-\alpha)\lambda_I)) + \lambda_I(\gamma_2 + \gamma_3 + \nu_1) - (\xi_D\gamma_3+c_p) \\
    \lambda_U'(t) = -\frac{\partial H}{\partial U} & = (\nu_1+\nu_2)\lambda_U - c_\nu \nu_1 - c_\omega \nu_2 \\
    \lambda_L'(t) = -\frac{\partial H}{\partial L} & = (\beta^1 A + \beta^2 I) (\lambda_L - \alpha \lambda_A - (1-\alpha)\lambda_I)
\end{align}

To find the optimal control, we compute
\begin{align}
    \frac{\partial H}{\partial \nu_1} & = - S \lambda_S - A\lambda_A - I\lambda_I - U\lambda_U + fS\lambda_L + c_\nu(S+A+U) \\
    \frac{\partial H}{\partial \nu_2} & = -S\lambda_S - U\lambda_U + fS\lambda_L + c_\nu S + c_\omega(S+A+U).
\end{align}
It is preferable to screen if
$$
\frac{\partial H}{\partial \nu_2} < \frac{\partial H}{\partial \nu_1}
$$
which is equivalent to
$$
    c_\nu (A+U) - c_\omega(S+A+U) > A\lambda_A + I\lambda_I.
$$
The terms on the right seem to be just an artifact of our (possibly faulty) assumptions
about screening people in groups A and I.

\subsection{Multiple cohorts}


\bibliographystyle{plain}
\bibliography{refs}

\end{document}
